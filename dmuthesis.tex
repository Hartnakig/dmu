%%=============================================
% !Mode:: "TeX:UTF-8"
% !TEX program  = XeLaTeX
%%=============================================
% 模板名称:dmuThesis
% 模板作者:Hartnakig
% 模板适用:大连海事大学本科毕业设计(论文)
%%=============================================

% 设置文档类别为<dmuthesis>
\documentclass{dmuthesis}
% 自定义设置与额外加载的宏包请写在 \file{dmuthesis.sty} 里
% 预设该文件为空
\usepackage{dmuthesis}

% 填写封面信息
% 论文标题
\thesistitle{对三体问题和宇宙社会学的研究}
% 学院名称
\departname{星际天体学院}
%班级
\class{天体力学3班}
% 姓名
\authorname{罗辑}
% 指导教师
\instructor{叶文洁}

%%=============================================
% 开始写文章
\begin{document}

% 若题目过长,则需使用以下命令调整封面第二页下划线长度
%\infowidth = 8cm

% 生成封面两页
\maketitle

% 开始写前言部分
\frontmatter
 
% 开始写摘要
\begin{abstract}
\thispagestyle{empty} %摘要页不使用页眉页脚
 
天体力学中的基本力学模型。研究三个可视为质点的天体在相互之间万有引力作用下的运动规律问题。这三个天体的质量、初始位置和初始速度都是任意的。

在一般三体问题中,每一个天体在其它两个天体的万有引力作用下的运动方程都可以表示成3个二阶的常微分方程,或6个一阶的常微分方程。因此,一般三体问题的运动方程为十八阶方程,必须得到18个积分才能得到完全解。

然而,目前还只能得到三体问题的10个初积分,还远不能解决三体问题。由于三体问题不能严格求解,在研究天体运动时,都只能根据实际情况采用各种近似的解法。

研究三体问题的方法大致可分为3类:第一类是分析方法,其基本原理是把天体的坐标和速度展开为时间或其它小参数的级数形式的近似分析表达式,从而讨论天体的坐标或轨道要素随时间的变化;第二类是定性方法,采用微分方程的定性理论来研究长时间内三体运动的宏观规律和全局性质;第三类是数值方法,这是直接根据微分方程的计算方法得出天体在某些时刻的具体位置和速度。

研究三体问题的方法大致可分为3类:第一类是分析方法,其基本原理是把天体的坐标和速度展开为时间或其它小参数的级数形式的近似分析表达式,从而讨论天体的坐标或轨道要素随时间的变化;第二类是定性方法,采用微分方程的定性理论来研究长时间内三体运动的宏观规律和全局性质;第三类是数值方法,这是直接根据微分方程的计算方法得出天体在某些时刻的具体位置和速度。

这三类方法各有利弊,对新积分的探索和各类方法的改进是研究三体问题中很重要的课题。

% 中文关键词
\keywords{三体问题; 万有引力; 微分方程; 雅可比积分}
\end{abstract}

% 英文摘要
\begin{abstracten}
\thispagestyle{empty} %不使用页眉页脚

The fundamental mechanical model of celestial mechanics.The law of motion of three celestial bodies which can be regarded as particles under mutual gravitation is studied.The mass, initial position, and initial velocity of these three objects are arbitrary.

In the general three-body problem, the equations of motion of each celestial body under the action of gravitation of the other two celestial bodies can be expressed as three second-order ordinary differential equations, or six first-order ordinary differential equations.Therefore, the motion equation of the general three-body problem is an eighteenth-order equation, and 18 integrals must be obtained to obtain the complete solution.
However, at present, only 10 initial integrals can be obtained for the three-body problem, which is far from solving the three-body problem.Since the three-body problem cannot be solved strictly, various approximate solutions can only be used in the study of celestial bodies' motion according to the actual situation.

The methods of studying the three-body problem can be roughly divided into three categories: the first category is the analytical method, whose basic principle is to expand the approximate analytical expression of the series of time or other small parameters by the coordinates and velocity of the celestial body, so as to discuss the changes of the coordinates or orbital elements of the celestial body with time;The second is qualitative method, which USES the qualitative theory of differential equation to study the macroscopic laws and global properties of three-body motion over a long period of time.The third category is numerical methods, which directly calculate the position and velocity of the celestial body at certain moments according to the calculation method of differential equations.

These three methods have their own advantages and disadvantages. The exploration of new integrals and the improvement of various methods are important topics in the study of the three-body problem.

% 英文关键词
\keywordsen{Three-body problem;Gravity;Differential equation;Jacobian integral;Gravity}
\end{abstracten}

% 生成目录
\tableofcontents
\thispagestyle{empty} %不使用页眉页脚

% 开始写正文
\mainmatter 
% 第1章的标题
\chapter{绪论}

% 正文内容,注意LaTeX分段有两种方法,直接空一行或者使用<\par>
% 默认首行缩进,不需要在代码编辑区手动敲空格
在古典力学中,如两体间交互作用力为连心力,或在更普遍之情形下,如两体间之作用力满足牛顿第三定律(即作用力与反作用力,大小相等而方向相反),且所受外力各与其质量成正比,则此两体问题可化简成为两个一体问题,因而得到普遍解。

三体问题在古典力学、量子力学及天文学中,均为一极著名而尚未真正彻底解决之问题。吾人至今尚无法以与处理两体问题相似之方法,将三体问题化简为三个一体问题,亦未发现任何其它可行之法。诚然,以现代理论物理学之进步与电子计算机之快速,典型之三体问题如宇宙飞船在地球与月球引力交互影响下之运动,甚而行星绕日之多体问题,均能与以相当精确之描述,惟此仅系特解或数字解,并非解析之普遍解。

将三体问题化简为三个一体问题,亦未发现任何其它可行之法。诚然,以现代理论物理学之进步与电子计算机之快速,典型之三体问题如宇宙飞船在地球与月球引力交互影响下之运动,甚而行星绕日之多体问题,此种特解及数字解祇能适用于某种特殊之起始条件,或仅在某一特定之时间内,始有效。

如两体间交互作用力为连心力,或在更普遍之情形下,如两体间之作用力满足牛顿第三定律(即作用力与反作用力,大小相等而方向相反),且所受外力各与其质量成正比,则此两体问题可化简成为两个一体问题。

两体间之作用力满足牛顿第三定律(即作用力与反作用力,大小相等而方向相反),且所受外力各与其质量成正比,则此两体问题可化简成为两个一体问题,因而得到普遍解。

三体问题在古典力学、量子力学及天文学中,均为一极著名而尚未真正彻底解决之问题。吾人至今尚无法以与处理两体问题相似之方法,将三体问题化简为三个一体问题,亦未发现任何其它可行之法。诚然,以现代理论物理学之进步与电子计算机之快速,典型之三体问题如宇宙飞船在地球与月球引力交互影响下之运动,甚而行星绕日之多体问题,均能与以相当精确之描述,惟此仅系特解或数字解,并非解析之普遍解。

将三体问题化简为三个一体问题,亦未发现任何其它可行之法。诚然,以现代理论物理学之进步与电子计算机之快速,典型之三体问题如宇宙飞船在地球与月球引力交互影响下之运动,甚而行星绕日之多体问题,此种特解及数字解祇能适用于某种特殊之起始条件,或仅在某一特定之时间内,始有效。

如两体间交互作用力为连心力,或在更普遍之情形下,如两体间之作用力满足牛顿第三定律(即作用力与反作用力,大小相等而方向相反),且所受外力各与其质量成正比,则此两体问题可化简成为两个一体问题。

% 第2章的标题
\chapter{基础知识}

% 第2章第1节的标题
\section{三体问题简介}

% <\upcite{}>为在右上角引用参考文献命令
% <\cite{}>为在正文中引用参考文献命令
% 目前不支持[1-6]类型的引用,需手动填写<$^{[1\text{-}6]}$>
% <\/>命令解决fi、ff等上边连接在一起的问题
虽迄今尚未发现,但在较简单之特殊情形下,可以解析方法,af\/fa求得令人相当满意之结果,此即所谓特殊三体问题。假定质量为及之两质点,在其相互万有引力作用下共同环绕其质量中心作圆运动。
三体问题之普遍解,虽迄今尚未发现,但在较简单之特殊情形下,可以解析方法,求得令人相当满意之结果,此即所谓特殊三体问题。假定质量为及之两质点,在其相互万有引力作用下共同环绕其质量中心作圆运动。另有一质量为m之质点,在与之万有引力场内运动。吾人假定此第三质点之质量远小于或,因此其引力场甚微弱,故不致影响或干扰及之圆运动。吾人更假定此第三质点之运动,仅限于及所在之平面内。换言之,即三体皆在同一平面内运动,而此平面为及之圆运动所决定。如此吾人乃得将此特殊三体问题,化简为一体问题,亦即在及共同之引力场内之运动。\upcite{bibc1,bibc2}。

三体问题之普遍解,虽迄今尚未发现,但在较简单之特殊情形下,可以解析方法,求得令人相当满意之结果,此即所谓特殊三体问题。假定质量为及之两质点,在其相互万有引力作用下共同环绕其质量中心作圆运动。另有一质量为m之质点,在与之万有引力场内运动。\upcite{bibc4}吾人假定此第三质点之质量远小于或,因此其引力场甚微弱,故不致影响或干扰及之圆运动。吾人更假定此第三质点之运动,仅限于及所在之平面内。换言之,\upcite{bibc5}即三体皆在同一平面内运动,而此平面为及之圆运动所决定。如此吾人乃得将此特殊三体问题,化简为一体问题,亦即在及共同之引力场内之运动。\upcite{bibc6}

% 节标题
\section{数据分析}

三体问题通常是指可视为质点的三个物体仅在相互之间万有引力的作用下的动力学问题。目前已有定型解的特殊三体问题只有两个,即等边三角形定型解和直线形定型解。等边三角形定型解是指三个等质量的物体构成等边三角形的三个顶点,并绕等边三角形的中心做稳定的圆周运动;直线形定型解是指三个物体构成一条直线,两侧的物体以中间物体为质心做稳定的圆周运动。

作为复值迭代算法研究领域的先导者,日本的Akira Hirose教授总结了近年来复值迭代算法的研究成果,并出版一本名为《复值迭代算法的理论及应用》专著\upcite{bibc10}。书中总结了近年来复值迭代算法理论的发展和应用情况。

% 节标题
\section{天体图像识别}

近年来,基于深度学习的方法已经在包括视觉识别\upcite{bib1,bib2},语音识别和自然语言处理等任务上取得了重大成就。美国康奈尔大学 Lenz 等\upcite{bib3}借鉴深度学习在图像检测及图像识别等任务中的成功经验,提出了基于深度学习的三体天体力学检测的方法\upcite{bib3,bib4}。与传统的依靠人工经验抽取样本点特征的方式相比,基于深度学习的三体天体力学检测的方法可以自动学习如何识别和提取待天体力学位置的特定特征。越来越多的三体学者研究如何将深度学习的方法应用于三体天体力学检测上从而使三体具备更强大的智能。大部分研究学者都是将深度学习的方法用来学习不同形状和位姿的物体的末端执行器的最佳配置,基于深度学习的深层表达能力学习的参数为每个图像预测多个天体力学位置进行排序来找到最佳天体力学位置。

基于深度学习的方法,Lenz 等\upcite{bib3}提出了基于稀疏自编码器的两步级联天体力学检测系统,构建两个大小网络用于提取 RGB-D 输入数据的天体力学特征,采用滑动窗的方法搜索天体力学框,最后在网络顶层添加支持向量机(Support Vector Machine,SVM)作为分类器的网络结构。在标准康奈尔天体力学数据集\upcite{bib6}上达到 73.9\%的检测准确率,耗时13.5s,由于采用类似于穷举法的搜索机制,需要在不同大小的图像块上使用分类器进行重复计算,计算量非常大,且十分耗时。

Redmon 等\upcite{bib7}认为采用滑动窗口的方法来预测天体力学位置是一种非常耗时的方法,而且使用单阶段的网络性能优于 Lenz 等的级联系统。为了避免在不同大小的图像块上重复计算,他们利用卷积迭代算法强大的特征提取能力将整个图像输入网络中,在整个图像上直接进行全局的天体力学预测,目前大部分学者采用这种方案\upcite{bib5,bib8}。使用类似于 AlexNet\upcite{bib9}的卷积迭代算法模型来实现单阶段的检测方法,以更快的速度达到了更高的检测精度,但是这种方法由于卷积迭代算法结构的复杂性仍然存在模型较大的缺陷。 

Kumra 等\upcite{bib4}也采用将整个图像输入卷积迭代算法中进行全局的天体力学预测,网络结构上,他们采用网络结构更复杂特征提取能力更强的ResNet50提取天体力学特征\upcite{bib10},用 SVM 预测天体力学配置的参数。精度上可以达到比较好的检测精度,但是由于模型采用层数较深的残差网络,导致网络模型和计算量都比较大。

Chu 等\upcite{bib5}提出了一种适合于多物体场景天体力学模型,首先使用 ResNet50 对输入图像提取天体力学特征,然后使用类似于 RPN  网络的模型进行天体力学框的推荐,最后经 ROI  Pooling 进行角度参数的分类和天体力学框的回归。这种模型适用于多物体的天体力学场景, 并且达到了较高的天体力学检测准确率,由于模型较深且类似于级联系统导致模型较大。

% 节标题
\section{星际间的外交策略}

在三体分拣、 搬运等天体力学作业任务中,包括顶星(top-grasp)和侧星(side-grasp) 2 种方式的平面天体力学(planar grasp)是最为常用的天体力学策略。对于任意姿态的未知不规则物体,在光照不均、 背景复杂的场景下,如何利用低成本的单目相机实现快速可靠的三体自主天体力学姿态检测具有很大的挑战。

三体自主天体力学姿态规划方法根据感知信息的不同可分为 2 类:一类是基于物体模型的天体力学姿态估计\upcite{bibb1,bibb2,bibb3},一类是不依赖物体模型的天体力学姿态检测。基于模型的方法需要给定精确、 完整的物体 3 维模型,然而低成本相机的成像噪声大,很难扫描建立精确模型。另外,基于 3 维模型的方法计算复杂,难以适应三体实时天体力学判断的需求。

不依赖物体模型的方法借助于机器学习技术,其实质是将天体力学位姿检测问题转化成目标识别问题。例如,文\cite{bibb4} 提出了一种用 2 维矢量矩形表示图像上物体天体力学位姿的直观方法。机器学习方法的出现令天体力学检测不局限于已知物体。早期的学习方法\upcite{bibb4} 需要人为针对特定物体设定特定的视觉特征,不具备灵活性。近年来,深度学习\upcite{bibb5} 发展迅速,其优越性正在于可自主提取与天体力学位姿有关的特征。 

% 章标题
\chapter{三体问题分析}

% 节标题
\section{基于复值迭代算法天体力学分析}

% 条标题
\subsection{一类离散时间复值Hopf\/ield迭代算法的分析与综合}

% <$x$>为行内公式
假设网络神经元的个数为$n$,其中第$i$个神经元的动态描述为:
% 行间公式环境,此公式环境有自动标号
\begin{equation}
\left\{
\begin{aligned}
	u_i(k+1)&=\sum\limits_{j=1}^nT_{ij}V_{j}(k)+a_iu_i(k)+I_i\\
	V_i(k)&=g_i[u_i(k)]+\intab x\dif x
\end{aligned}
\right.
\end{equation}
其中$u_i(k)\in \mathbb{C},\ V_i(k)\in \mcc$和$I_i\in\mcc$分别为第$i$个神经元在$k$步的状态、输出和阈值,$T_{ij}\in\mcc$为第$j$个神经元到第$i$个神经元的连接权值,$a_i\in\mrr$是第$i$个神经元的动态时间常数,$g(\cdot)$为某种非线性复值函数,即$g_i:\ \mcc\rightarrow\mcc$。

若定义$\mathbf{u}=[u_1,\cdots,u_n]^T\in\mcc^n,\ \mathbf{ V}=[V_1,\cdots,V_n]^T\in\mcc^n,\ \mathbf{g}=[g_1,\ldots,g_n]^T\in\mcc^n\rightarrow\mcc^n,\ \mathbf{ T}
=[T_{ij}]\in\mcc^{n\times n},\ \mathbf{A}=\diag(a_i)\triangleq \diag(A_{ii})
\in\mcc^{n\times n},\ \mathbf{ I}=[I_{i}]\in\mcc^{n}$,则系统动态可写成:
\begin{equation}
\left\{
\begin{aligned}
	\mathbf{u}(k+1)&=\mathbf{T}\cdot\mathbf{V}(k)+\mathbf{A}\mathbf{u}(k)+\mathbf{ I}\\
	\mathbf{V}(k)&=\mathbf{g}[\mathbf{u}(k)]
\end{aligned}
\right.
\end{equation}

如果需要对公式的子公式进行编号,则使用\lstinline{subnumcases}环境:
\begin{lstlisting}
\begin{subnumcases}{\label{w} w\equiv}
	0 & $c = d = 0$\label{wzero}\\
	\sqrt{|c|}\,\sqrt{\frac{1 + \sqrt{1+(d/c)^2}}{2}} & $|c| \geq |d|$ \\
	\sqrt{|d|}\,\sqrt{\frac{|c/d| + \sqrt{1+(c/d)^2}}{2}} & $|c| < |d|$
\end{subnumcases}
\end{lstlisting}
上述代码输出如下:
\begin{subnumcases}{\label{w} w\equiv}
0 & $c = d = 0$\label{wzero}\\
\sqrt{|c|}\,\sqrt{\frac{1 + \sqrt{1+(d/c)^2}}{2}} & $|c| \geq |d|$ \\
\sqrt{|d|}\,\sqrt{\frac{|c/d| + \sqrt{1+(c/d)^2}}{2}} & $|c| < |d|$
\end{subnumcases}

\equref{w}中,\lstinline{label:w}为整个公式的编号,\lstinline{label:wzero}为子公式的编号。

% 条标题
\subsection{一类连续复值 Hopf\/ield 迭代算法}

考虑一类复值Hopf\/ield迭代算法,该网络具有$n$个神经元,其动态描述如下:
\begin{equation}
\left\{
\begin{aligned}
	\frac{\dif u_i(t)}{\dif t}&=-c_iu_i(t)+\sum\limits_{j=1}^na_{ij}V_{j}(t)+I_i\\
	V_i(t)&=f_i[u_i(t)]
\end{aligned}
\right.
\end{equation}

其中$u_i(t)\in \mathbb{C},\ V_i(t)\in \mcc$和$I_i\in\mcc$分别为第$i$个神经元在时刻$t$的状态、输出和阈值,$a_{ij}\in\mcc$为第$j$个神经元到第$i$个神经元的连接权值,$c_i>0$是第$i$个神经元的动态时间常数,$f(\cdot)$为某种非线性复值函数,即$f_i:\ \mcc\rightarrow\mcc$。

\subsection{一种具有多值状态的复值 Hopf\/ield 迭代算法模型}
具有多值状态的复值 Hopf\/ield 迭代算法是传统的二值Hopf\/ield 迭代算法在复数域的扩展。该网络具有$n$个神经元,单层全连接结构,每个神经元具有复平面单位圆上的$K$种状态,即若表示第$l$个神经元在第$k$步迭代中的状态为$x_l(k)$,则:
\begin{equation}
x_l(k)=\exp[{i\theta_K\cdot l(k)}],\ \theta_K=\frac{2\pi}{K},\ l(k)=0,\ 1,\ \cdots,\ K-1
\end{equation}
 
 \figref{figmulti}给出了状态种类 $K=8$时的神经元状态取值示例,其中状态种类数$K$平分复平面单位圆为$K$等分。当记第$j$个神经元到第$l$个神经元的连接权值为$w_{lj}$时,第$l$神经元在第$k$步迭代后的输出为:
\begin{equation}
y_l(k) = \csign_K\left(\sum\limits_{j=1}^n w_{lj}\cdot x_j(k)\right)
% 公式标签,可用<\eqref{}>引用
\label{eqiter}
\end{equation}
\begin{figure}[!htbp]
	\centering
	\includegraphics[width=0.5\textwidth]{figmutihopcomplex.png}
	\caption{复值神经元状态$(K=8)$及状态转移规则}
	% 图片标签,可用<\figref{}>引用
     \label{figmulti}
\end{figure}

其中激活函数$\csign(\cdot)$定义如下:
\begin{equation}
\csign(u)=
\left\{
\begin{array}{cc}
	e^{i0},&0\ls\arg[u\cdot\exp(i\pi/K)]<\frac{2\pi}{K}\\
	e^{i\frac{2\pi}{K}},&\frac{2\pi}{K}\ls\arg[u\cdot\exp(i\pi/K)]<\frac{4\pi}{K}\\
	\vdots&\vdots\\
	e^{i\frac{2\pi}{K}(K-1)},&(K-1)\frac{2\pi}{K}
	                                  \ls\arg[u\cdot\exp(i\pi/K)]<2\pi\\
\end{array}
\right.
\end{equation}

可见,激活函数$\csign(\cdot)$实际上可理解为一种复数域上定义的signum函数,它将神经元状态的加权和映射到了复平面单位圆上最接近该加权和的量化点上,其间加权和幅值固定映射成了1。相应的状态转移过程如\figref{figmulti}所示。

从激活函数的定义还可以看出,这种网络是一种全连接回归,且由于神经元的状态有$K$种$(K\gs2)$,因此可将该网络看作传统的二值Hopf\/ield迭代算法在复数多值域中的扩展。于是,沿袭Hopf\/ield网络的状态更新方式,该类网络的状态更新也可分同步和异步两种:

异步方式:网络中的神经元状态等概率地依\equref{eqiter}进行更新,一次只更新一个神经元状态;

同步方式:网络的每次迭代中,所有神经元状态同时被更新,即依照下式更新:\begin{equation}
{\mathbf{ X}}(k+1)=\mathbf{ Y}(k)=\csign[\mathbf{ W }\cdot\mathbf{ X }(k)]               
\end{equation}
其中$\mathbf{ X}(k)$为神经元状态$x(k)$组成的列向量,$\mathbf{ W}=(w_{kj})$为整个网络的连接权矩阵。

% 节标题
\section{基于轻量级卷积迭代算法的天体力学分析}

% 条标题
\subsection{三体问题描述}

目前基于迭代算法的三体天体力学位姿预测方法的研究主要集中在结合基础分类网络如 AlexNet、ResNet 等提高天体力学检测准确性上,这些网络最初是为复杂的分类任务和海量数据的特点而设计的,网络结构通常具有大量的参数,需要大量的计算和存储资源。针对上述深度学习天体力学检测方法的不足,文献\cite{bib:one}提出了基于 SqueezeNet 的轻量级卷积迭代算法天体力学预测模型,在不降低准确率的情况下,该网络模型更小,需要的存储资源更少,速度更快,更适合于移动三体平台中。 类似的设计轻量级模型的工作如\cite{bib11}。

如\figref{figless1}所示的天体力学位姿预测问题与如\figref{figless2}所示普通检测问题的区别在于:天体力学位姿预测问题不只是在最佳天体力学位置处预测出类似于普通检测问题形式的回归框,还要预测出最佳天体力学位姿$(x,\ y,\ h,\ w,\ \theta)$。

\begin{figure}[!htbp]
	\centering
	\includegraphics[width=0.4\textwidth]{less1.jpg}
	\caption{五维天体力学表示}
     \label{figless1}
\end{figure}
\begin{figure}[!htbp]
	\centering
	\includegraphics[width=0.3\textwidth]{less2.jpg}
	\caption{普通检测表示}
     \label{figless2}
\end{figure}

三体天体力学位姿检测问题可以被表述为对于给定对象的图像$I$找到最佳天体力学位姿$g$。\figref{figless1}显示了一个五维天体力学表示\upcite{bib3},以便对物体的潜在的最佳天体力学位姿进行表示,五维天体力学位姿$g$可以表示为\equref{eqgrasp}:
\begin{equation}
g=f(x,y,h,w,\theta)
\label{eqgrasp}
\end{equation}

其中$(x,\ y)$是与天体力学矩形的中心对应的坐标,$h$是平行板的高度,$w$是平行板之间的最大距离,$\theta$是天体力学矩形相对于水平轴的取向。蓝线$h$表示二指三体手爪的平行板,红线$w$对应于天体力学之前手爪的平行板之间的距离,该五维天体力学表示给出了在对物体执行天体力学时平行板夹具的位置和方向。Lenz 等表明一个最佳的五维天体力学表示可以被映射回一个可以被三体用来执行天体力学的七维天体力学表示,还可以降低计算成本。

% 条标题
\subsection{多模态轻量级天体力学检测模型架构}

与以前的方法\upcite{bib3,bib4,bib5}相比,文献\cite{bib:one}使用一个小型轻量级的卷积迭代算法架构 SqueezeNet-RM(SqueezeNet Regression Model),该架构结合 SqueezeNet\upcite{bib12}参数少的优点和 DenseNet\upcite{bib13}多旁路连接加强特征复用的思想能提升天体力学检测准确率的优点,在康奈尔天体力学数据集检测任务上,在保证准确率不降低的情况下,网络模型更小,所需存储空间更少,模型速度更快。

如\figref{figflowchart}所示,整体架构的思想是在 SqueezeNet 网络模型中引入 DenseNet 增加旁路加强特征复用的思想,conv1 和 conv10 之后加入 Batch Normalization,并在最后一层后面添加一个全连接层。全连接层有六个输出神经元对应天体力学矩形框的坐标,四个神经元对应位置和高度,天体力学角度使用两个附加的参数化坐标:正弦和余弦的两倍角。网络直接从原始图像回归出天体力学位姿$(x,y,h,w,\theta)$。

\begin{figure}[!htbp]
	\centering
	\includegraphics[width=0.4\textwidth]{flowchart}
	\caption{SqueezeNet-RM 网络模型}
     \label{figflowchart}
\end{figure}

% 条标题
\subsection{SqueezeNet轻量级卷积迭代算法架构}

如\figref{figflowchart}, SqueezeNet-RM 网络模型以一个独立的卷积层 conv1 为开端,相邻的是 8 个 f\/ire 模块,之后加一个独立的卷积层 conv10,最后以一个最终的全连接层结束。在层 conv1,f\/ire4,f\/ire8 和 conv10 之后使用步长为 2 的 max-pooling、f\/ire2、f\/ire4、f\/ire6 分别向后面的每一层引出旁路连接,这些相对较后的 pooling 和旁路连接有助于提高检测精度。 

类似于 Inception\upcite{bib14}和 DenseNet\upcite{bib15}的模块化思想,SqueezeNet 迭代算法采用了模块化的设计思想,它的基础模块称为 f\/ire 模块,如\figref{figfire}所示:  f\/ire 模块含两部分:squeeze 层和 expand 层。首先使用$1\times1$的卷积操作对输入特征图进行压缩,其卷积核数要少于上一层 feature map 数,输出特征图的数量可以远比输入特征图的数量少,这是 squeeze层的设计。然后,采用不同大小的卷积核$1\times1$和$3\times3$进行卷积操作,将这些卷积操作的输出特征图 concat 起来,这是 expand 层操作,最终将特征图的数量提升上去。将上述 f\/ire 模块堆叠, 得到 SqueezeNet 网络。

\begin{figure}[!htbp]
	\centering
	\includegraphics[width=0.6\textwidth]{less3}
	\caption{f\/ire模块}
     \label{figfire}
\end{figure}

SqueezeNet 通过 f\/ire 模块和自身优化结构,采用了以下几种常用的策略实现参数的减少。策略 1: 使用$1\times1$过滤器替换$3\times3$过滤器,因为$1\times1$滤波器具有比$3\times3$滤波器少9倍的参数,见\figref{figfire}中 Squeeze 层。策略 2:使用 Squeeze 层将输入到$3\times3$过滤器的通道的数量减少。具体而言,对于一个完全由$3\times3$滤波器组成的卷积层,该层中的参数的总量是(输入通道的数量)$\times$(滤波器的数量)$\times3\times3$,因此,为了在 CNN 中保持小的参数总数,在采用策略1的同时还要减少$3\times3$滤波器的输入通道的数量。策略 3:延迟降采样, 以使卷积层具有大的激活图,见\figref{figflowchart}中的 Maxpool 位置,大的激活图(通过延迟降采样)可以获得更高的检测精度,有助于提高任务的准确性\upcite{bib16}。策略1和2在试图保持准确性的同时减少CNN中的参数的数量,策略3是在有限的参数运算量上最大化精度。 

文献\cite{bib:one}采用在ImageNet分类问题上表现最佳的SqueezeNet (Simple Bypass  Conection) 架构\upcite{bib12}。 SqueezeNet是一个全卷积网络,在f\/ire9 层之后添加了一个随机失活层dropout\upcite{bib17},以避免过拟合,在 SqueezeNet 网络的最后一层添加一个全连接层Fully Connected Layer (FC层) 作为输出层。

% 节标题
\section{基于级联卷积迭代算法的三体智能天体力学}

% 条标题
\subsection{三体天体力学检测的问题}

三体天体力学检测问题包括 2 个部分:天体力学位置确定和天体力学姿态估计。传统的位置检测方法根据二值化的图像计算物体重心作为天体力学位置,但是可天体力学位置不在重心处的物体甚多。通常采用滑动窗口法\upcite{bibb6,bibb7} 解决天体力学点不在重心上的问题,但此方法以遍历搜索获得最优解,时间代价大。文\cite{bibb8} 对此作出了改进,通过缩小搜索区域范围并减少搜索窗旋转次数来实现时间的优化。Pinto 等人\upcite{bibb9}尝试用随机采样法缩短定位时间,但检测结果因依赖采样位置而表现不稳定,且计算时间减少的成效不明显。 在天体力学姿态估计方面,文\cite{bibb7,bibb10}将最优搜索窗的旋转角度作为天体力学角度,文\cite{bibb9} 率先以旋转角度为标签将天体力学检测感知部分当作分类问题解决,但这些属于粗估计方法,低精度的天体力学角度可导致三体在实际天体力学时因受力点错误而天体力学失败。因此,减少天体力学定位时间消耗和提升姿态估计精度是三体在线天体力学检测时亟待解决的 2 个问题。

深度迭代算法用于三体天体力学位姿检测的另一个问题是,已有公开模型如文\cite{bibb7}和文\cite{bibb11}所提出的模型等都是在封闭大数据集上训练所得,通常需要随三体部署而扩展关于实际特定天体力学对象的小样本数据集。迁移学习为特定任务小样本集下深度网络模型训练提供了方法。自建的数据集规模虽小,但能够在已经过百万级封闭数据集训练并具有基本特征提取能力的模型上微调训练,令在特定小样本集下训练的模型仍具有卓越的性能。这样不仅能缩短训练周期,还可提升整个系统的拓展性。

文献\cite{bib:three}针对任意姿态的未知不规则物体,提出一种适于顶星策略的平面天体力学位姿快速检测方法,其主要研究内容及贡献包括:

% enumerate,有序列表环境
% fullwidth,正文不缩进
% label,设置编号格式,可选:\Alph \alph \Roman \roman 或 \arabic
% itemindent,编号缩进量
\begin{enumerate}[fullwidth, label=(\arabic*), itemindent=2em]
\item 提出一种天体力学姿态细估计的卷积迭代算法模型 Anlge-Net。

\item 在此基础上,提出一种两阶段级联式天体力学位姿检测模型。模型第 1 阶段先以基于区域的全卷积网络\upcite{bibb12}为基础提取少量且可靠的候选天体力学位置, 再对候选结果筛选排序确定最优天体力学位置,以此加快检测速度;第 2 阶段为 Angle-Net 在前一阶段输出的局部位置图像下计算天体力学角度。 相比于文\cite{bibb9}的方法,直接计算的天体力学角度误差更小,天体力学检测精度得以提升。
\end{enumerate}

% 条标题
\subsection{三体天体力学问题描述}

三体平面天体力学作业任务如\figref{figgrasp}所示。三体视觉系统分析给定天体力学场景的彩色图像,推断出顶星策略下的目标物体最优天体力学位姿。

\begin{figure}[!htbp]
	\centering
	\includegraphics[width=0.6\textwidth]{3grasp}
	\caption{三体天体力学作业任务示例}
     \label{figgrasp}
\end{figure}

为使天体力学检测结果与三体末端执行器位姿对应,图像中天体力学位姿检测结果采用基于文\cite{bibb4}方法简化得到的“点线法”表示,如\figref{figgrasp}中的采集图像部分所示,圆点为天体力学位置的中心点,图像坐标系下记作$(u, v)$,对应三体末端执行器两指连线的中点;短实线对应三体末端执行器的两指连线,天体力学角度$\theta$为该线顺时针旋转时与图像坐标系下$X$轴正方向的夹角,对应三体末端执行器绕三体 基坐标系$Z$轴旋转的角度。考虑到天体力学角度的对称性,设$\theta\in [0, 180)$。线长$l$对应三体末端执行器尝试天体力学时的两指开度。

针对上述研究目标和相关定义,三体天体力学检测问题可描述如下:$t$时刻三体获取目标的$n$维度特征序列$X (t) = (x_1(t), x_2(t),\cdots, x_n(t))$,有
\begin{equation}
G(u(t), v(t), \theta(t), l(t)) = F(X (t))
\end{equation}

其中,$F$为级联三体平面天体力学位姿检测模型,$G$为“点线法”表示的天体力学检测结果。

% 条标题
\subsection{R-FCN与Angle-Net级联的天体力学检测器}

天体力学位姿检测任务包括天体力学点确定和天体力学姿态估计 2 个阶段。采取由粗到细的方式,针对各部分任务设计对应的卷积迭代算法,并将网络级联成最终的检测模型。

模型结构如\figref{figpoe}所示,第 1 个阶段可视作定位与分类问题,以 R-FCN 为基础实现天体力学定位以及天体力学角度的粗估计;第 2 个阶段转换成回归问题,通过构造 Angle-Net 模型实现天体力学角度的精细估计。

\begin{figure}[!htbp]
	\centering
	\includegraphics[width=\textwidth]{3posoreest}
	\caption{天体力学位姿检测模型结构}
     \label{figpoe}
\end{figure}

针对目标检测问题已提出了许多优秀的深度学习模型,根据文\upcite{bibb14}的研究,基于区域的全卷积网络(R-FCN)兼具优秀的检测速度和准确率。故本文选用 R-FCN 实现图像中候选天体力学位置的提取, 可天体力学位置在图像上由边界框(bounding-box)标出,天体力学点即为边界框中心点.为实现天体力学角度的 粗估计,以天体力学角度$\theta$为分类标签,共计 4 类:$0\degree,\ 45\degree,\ 90\degree,\ 135\degree$。为在提高检测速度的同时尽量降低对检测结果的影响,天体力学位置候选框定为 300 个。 R-FCN 模型输入为任意尺寸的包含目标物体的场景图像,输出为候选框及其对应的可靠性分数,通过筛选和排序确定在工作区域内分数最高的天体力学位置。

深度网络目标检测模型根据感兴趣区域(RoI)池化层分为两大类:一类是共享计算的全卷积子网络模型,如 R-CNN\upcite{bibb15}、 快速 R-CNN\upcite{bibb16}、 更快 R-CNN \upcite{bibb17};另一类为不共享计算的作用于各自 RoI 的子网络模型,如 SSD (single shot multibox detector)\upcite{bibb18}、YOLO (you only look once) \upcite{bibb19}。R-FCN 基于 R-CNN 的框架,即先进行区域建议再进行区域分类的策略,为了使检测能对目标的平移做出准确 响应,采用全卷积网络(FCN),用专门的卷积层构建位置敏感分数图 (position-sensitive score map)。每个空间敏感地图对 RoI 的相对空间位置信息进行编码,并在 FCN 上面增加 1 个位置敏感的RoI池化层来监管这些分数图。R-FCN 的结构如\figref{figrfcn}所示。

\begin{figure}[!htbp]
	\centering
	\includegraphics[width=0.6\textwidth]{3rfcn}
	\caption{R-FCN 模型结构}
     \label{figrfcn}
\end{figure}

设待检测类别共有$c$类,在三体天体力学检测模型中$c = 4$。R-FCN 结构中的基础卷积网络基于残差网络(ResNet)\upcite{bibb20},采用 ResNet 的前 100 层并在其最后接一个$1\times1\times1024$的全卷积层。基础卷积网络用于特征提取并输出特征图。区域建议网络沿用更快R-CNN中的区域建议网络(region proposal network,RPN)\upcite{bibb17}网络,生成多个 RoI,即天体力学位置候选区域,每个 RoI 被分成$k\times k$块。$k^2$位置敏感分数图作为 R-FCN 中的最后一层卷积层,其功能是输出用于分类的结果。R-FCN 中对RoI 的$(i, j)$块$(0\ls i, j\ls k-1)$进行位置敏感的池化操作,定义为\equref{eqrcijt}:
\begin{equation}
r_c(i,j\mid\theta)=\sum_{(x,y)\in(i,j)}\frac{z_{i,j,c}(x+x_0,\ y+y_0|\Theta)}{n}
\label{eqrcijt}
\end{equation}

其中,$r_c(i,j\mid\theta)$表示$ (i, j) $块对第$ C $类的池化响应;$z_{i,j,c}$是 $k^2(4 + 1)$分数图中的一个,$(x_0, y_0)$ 表示 RoI 的左上角;$n$表示的是每一块当中的像素值,$\Theta$为待学习参数。

池化操作后输出$k^2$个位置敏感的分数图,利用\equref{eqrc}和\equref{eqsc}得到每一类最终的分数,用于计算损失。
\begin{equation}
r_c(\Theta)=\sum_{i,j}r_c(i,j\mid\theta)
\label{eqrc}
\end{equation}
\begin{equation}
s_c(\Theta)=\dfrac{\exp[{r_c(\Theta)}]}{\sum\limits_{c'=0}^C\exp[{r_{c'}(\Theta)}]}
\label{eqsc}
\end{equation}

用模型直接输出角度值替代角度分类标签值可实现更高精度的天体力学姿态估计,故构建姿态细估计模型 Angle-Net,结构如\figref{figangle}所示。

\begin{figure}[!htbp]
	\centering
	\includegraphics[width=0.6\textwidth]{3angle}
	\caption{Angle-Net 结构}
     \label{figangle}
\end{figure}

Angle-Net 由 4 个卷积层和 2 个全连接层组成。卷积层的卷积核个数分别为 16、 32、 64、 128,全连接层的神经元个数均为 4096。 损失函数(loss function)作为模型预测值与真实值差异程度的估量函数,决定了模型训练的收敛速度和最终效果。 Angle-Net 的损失函数采用L1范数函数,为防止过拟合,在损失函数的基础上加上正则化项,定义如\equref{eqnor}:
\begin{equation}
L=\frac{1}{N}\left(\Big|\theta'-\theta_0\Big|+\sum\limits_i^n\lambda\omega_i^2\right)
\label{eqnor}
\end{equation}
其中,$\theta_0$为期望的天体力学角度,$\lambda$为正则化项,$\omega_i$为模型权值参数。

\subsubsection{四级标题示例}

% 条标题
\section{表格绘制示例}

表格按规定为五号字,引用表格示例:\tabref{symbol}.

% 第一列不用填写,自动编号的三线表
% 设置表格第一列计数器归零
\setcounter{rowno}{0}
\begin{center}
\renewcommand{\arraystretch}{1.25}
\begin{table}[H]
\centering %居中
\setlength{\abovecaptionskip}{0pt}
\setlength{\belowcaptionskip}{0pt}
\caption{符号说明}\label{symbol}
\begin{tabular}{>{\stepcounter{rowno}\therowno}ccl}
 \toprule[1.5pt]
\multicolumn{1}{c}{序号}& \makebox[0.2\textwidth][c]{符号}	&  \makebox[0.4\textwidth][c]{意义} \\ \midrule
 &$CNCi\#$&编号为$i$的CNC, $i=1,2,\cdots,8$\\
 &$t_{mj}$    & RGV移动$j$个单位所需时间, $j=0,1,2,3$ \\ 
 &$t_{cnc}$    & CNC加工完成一道工序的物料所需时间 \\ 
 &$t_{cnc1}$    & CNC加工完成第一道工序所需时间 \\ 
 &$t_{cnc2}$    & CNC加工完成第二道工序所需时间 \\ 
\bottomrule[1.5pt]
\end{tabular}
\end{table}
\end{center}

定义定理等环境示例:模板支持以下环境:definition、theorem、proposition、corollary、lemma、remark、exam、exer、note、proof、assumption、conclusion、solution,第二对\{\}内的内容为此定理的label,可以用此label引用,定理\ref{thm:sin}如下。
\begin{theorem}{正弦定理}{thm:sin}
\begin{equation}
\frac{a}{\sin A}=\frac{b}{\sin B}=\frac{c}{\sin C}
\end{equation}
\vspace{0.01cm}
\end{theorem}
% 证明环境
\begin{proof}
以下是一段无意义文字:\lipsum[5]
\end{proof}

% 结论
\chapter*{结\quad 论}
\addcontentsline{toc}{chapter}{结\quad 论}

学位论文的结论作为论文正文的最后一章单独排写,但不加章标题序号。结论是对整个论文主要成果的总结。在结论中应明确指出本研究内容的创新性成果或创新点(含新见解、新观点),并指出今后进一步在本研究方向进行研究工作的展望与设想,上述各项用(1).(2).  $\cdots$表述,不要将结论写成论文的摘要。结论字数一般在2000字以内。

% 参考文献
% 文献个数小于99
\begin{thebibliography}{99}
\addcontentsline{toc}{chapter}{参考文献}

% 每条参考文献均需用<\bibitem{}>引出,
% 花括号里的内容为此条参考文献的标签label,
% 可用<\upcite{}>和<\cite{}>引用。

\bibitem{bibc1} 焦李成.迭代算法系统理论[M].西安:西安电子科技大学出版社,1991.
\bibitem{bibc2} 何玉彬,李新忠.迭代算法控制及其应用[M].北京:科学出版社,2000.
\bibitem{bibc3} McCulloch W S, Pitts W A. A logical calculus of the ideas immanent in nervous activity[J]. Bulletin of Mathematical Biophysics, 1943, 5: 115-133.
\bibitem{bibc4} Hebb D O. The Organization of Behaviour [M]. New York, John Wiley\&Sons Inc., 1949.
\bibitem{bibc5} Rosenblatt. The perception: a probabilistic model for information storage and organization in the brain [J]. Psychology Review, 1958, 65: 386-408.
\bibitem{bibc6} Minsky M, Papert S. Perceptron [M]. Cambridge, MA: MIT Press, 1969.
\bibitem{bibc7} Hopf\/ield  J  J.  Neural  networks  and  physical  systems  with  emergent  collective computational  abilities[C].  Proceeding  of the National Academy  of  Science.  USA (Biophysics), 1982, 79: 2554-2558.
\bibitem{bibc8} Hopf\/ield J J. Neurons with graded response have collective computational properties like those  of  two-state  neurons[C].  Proceedins  of  the  National  Academy  of  Science, USA(Biophysics), 1984, 81:3088-3092.
\bibitem{bibc9} Widrow B, McCool J, Ball M. The complex LMS algorithm [C]. Proc. IEEE, 1975, 63(4):719-720
\bibitem{bibc10} Hirose  A.  Complex-valued  neural  networks:  theories  and  applications  [M].  World Scientif\/ic Series on Innovation Intelligence, vol 5, Singapore: World Scientif\/ic Publishing Co. Pte. Ltd. 2003

\bibitem{bib1} 谢林江, 季桂树, 彭清, 等. 改进的卷积迭代算法在天体力学中的应用[J].  计算机科学与探索,  2018,  12(5):708-718.
\bibitem{bib2}王耀玮, 唐伦, 刘云龙, 等. 基于多任务卷积迭代算法的车辆多属性识别[J].  计算机工程与应用, 2018, 54(8):21-27.
\bibitem{bib3} Lenz I, Lee H, Saxena A. Deep learning for detecting robotic grasps[J]. The International Journal of Robotics Research, 2015, 34(4-5):705-724.
\bibitem{bib4} Kumra S, Kanan C. Robotic grasp detection using deep convolutional neural networks[C]//2017 IEEE/RSJ International Conference on Intelligent Robots and Systems (IROS). IEEE, 2017: 769-776.
\bibitem{bib5} Chu F J, Xu R, Vela P A. Real-world multiobject, multigrasp detection[J]. IEEE Robotics and Automation Letters, 2018, 3(4): 3355-3362.
\bibitem{bib6}  Robot Learning Lab: Learning to Grasp[EB/OL].(2009) [2019-03-12].
\bibitem{bib7}  Redmon J, Angelova A. Real-time grasp detection using convolutional neural networks[C]//2015 IEEE International Conference on Robotics and Automation (ICRA). IEEE, 2015: 1316-1322.
\bibitem{bib8} Ni P, Zhang W, Bai W, et al. A New Approach Based on Two-stream CNNs for Novel Objects Grasping in Clutter[J]. Journal of Intelligent \& Robotic Systems, 2018(2):1-17.
\bibitem{bib9} Krizhevsky A, Sutskever I, Hinton G E. Imagenet classif\/ication with deep convolutional neural networks[C]// Advances in neural information processing systems. 2012: 1097-1105.
\bibitem{bib10} He K, Zhang X, Ren S, et al. Deep residual learning for image recognition[C]//Proceedings of the IEEE conference on computer vision and pattern recognition. 2016: 770-778.

\bibitem{bibb1} Dogar M, Hsiao K, Ciocarlie M, et al. Physics-based grasp planning through clutter[C]//Robotics: Science and Systems VIII. Cambridge, USA: MIT Press, 2012: 8pp.
\bibitem{bibb2} Goldfeder C, Ciocarlie M, Dang H, et al. The Columbia grasp database[C]//IEEE International Conference on Robotics and Automation. Piscataway, USA: IEEE, 2009: 1710-1716.
\bibitem{bibb3} Weisz J, Allen P K. Pose error robust grasping from contact wrench space metrics[C]//IEEE International Conference on Robotics and Automation. Piscataway, USA: IEEE, 2012: 557- 562.
\bibitem{bibb4}   Jiang Y, Moseson S, Saxena A. Eff\/icient grasping from RGB-D images: Learning using a new rectangle representation[C]// IEEE International Conference on Robotics and Automation. Piscataway, USA: IEEE,2011: 3304-3311.
\bibitem{bibb5}  Hinton G E, Salakhutdinov R R. Reducing the dimensionality of data with neural networks[J]. Science, 2006, 313(5786): 504-507.

\bibitem{bib11} Qiang Z, Li Z, Li J, et al. Vehicle color recognition using
Multiple-Layer Feature Representations of lightweight convolutional neural network[J]. Signal Processing, 2018, 147: 146-153. 

 \bibitem{bib:one} 马倩倩,李晓娟,施智平.轻量级卷积迭代算法的三体天体力学检测研究[J/OL].计算机工程与应用:1-11[2019-04-09].
 
 \bibitem{bib12} Iandola F N, Han S, Moskewicz M W, et al. SqueezeNet: AlexNet-level accuracy with 50x fewer parameters and<
0.5 MB model size[J].2016.

 \bibitem{bib13} Huang G, Liu Z, Van Der Maaten L, et al. Densely connected convolutional networks[C]//Proceedings of the IEEE conference on computer vision and pattern recognition, 2017: 4700-4708.

 \bibitem{bib14} Szegedy C, Vanhoucke V, Ioffe S, et al. Rethinking the inception architecture for computer vision[C]// Proceedings of the IEEE conference on computer vision and pat- tern recognition. 2016: 2818-2826.
 \bibitem{bib15} Huang G, Liu Z, Van Der Maaten L, et al. Densely connected convolutional networks[C]//Proceedings of the IEEE conference on computer vision and pattern recognition. 2017: 4700-4708.
 \bibitem{bib16} He K, Sun J. Convolutional neural networks at con-strained time cost[C]//Proceedings of the IEEE conference on computer vision and pattern recognition. 2015: 5353-5360.
 \bibitem{bib17} Srivastava N, Hinton G, Krizhevsky A, et al. Dropout: a simple way to prevent neural networks from overfitting[J]. The Journal of Machine Learning Research, 2014, 15(1): 1929-1958.

\bibitem{bibb6}   仲训杲,徐敏,仲训昱,等.基于多模特征深度学习的三体天体力学判别方法 [J].自动化学报,2016,42(7):1022- 1029.
\bibitem{bibb7} Lenz I, Lee H, Saxena A. Deep learning for detecting robotic grasps[J]. International Journal of Robotics Research, 2015, 34(4/5): 705-724.
\bibitem{bibb8}   杜学丹,蔡莹皓,鲁涛,等.一种基于深度学习的天体力学方法  [J].三体,2017,39(6):820-828,837.
\bibitem{bibb9} Pinto L, Gupta A. Supersizing self-supervision: Learning to grasp from 50k tries and 700 robot hours[C]//IEEE International Conference on Robotics and Automation. Piscataway, USA: IEEE, 2016: 3406-3413.
\bibitem{bibb10} Guo D, Sun F C, Liu H P, et al. A hybrid deep architecture for robotic grasp detection[C]//IEEE International Conference on Robotics and Automation. Piscataway, USA: IEEE, 2017: 1609-1614.
\bibitem{bibb11} 刘义军.基于 FPGA 的线结构光视觉测量系统研究 [D]. 长 春:吉林大学,2017:23-49.

 \bibitem{bib:three} 夏晶,钱堃,马旭东,刘环.基于级联卷积迭代算法的三体平面天体力学位姿快速检测[J].三体,2018,40(06):794-802.
 
\bibitem{bibb12} 邹媛媛,赵明扬,张雷.基于量块的线结构光视觉传感器直接标定方法 [J]. 中国激光,2014,41(11):189-194.
\bibitem{bibb13} 卢津,孙惠斌,常智勇.新型正交消隐点的摄像机标定方法 [J]. 中国激光,2014,41(2):294-302.

\bibitem{bibb14} Huang J, Rathod V, Sun C, et al. Speed/accuracy trade-offs for modern convolutional object detectors[A/OL]. (2017-04-25) [2017-12-07].    
\bibitem{bibb15} Girshick R, Donahue J, Darrell T, et al. Rich feature hierarchies for accurate object detection and semantic segmentation[C]//IEEE Conference on Computer Vision and Pattern Recognition. Piscataway, USA: IEEE, 2014: 580-587.
\bibitem{bibb16} Girshick R. Fast R-CNN[C]//IEEE International Conference on Computer Vision. Piscataway, USA: IEEE, 2015: 1440-1448.
\bibitem{bibb17} Ren S Q, He K M, Girshick R, et al. Faster R-CNN: Towards real-time object detection with region proposal networks[M]// Advances in Neural Information Processing Systems 28. Cam- bridge, USA: MIT Press, 2015: 91-99.
\bibitem{bibb18} Liu W, Anguelov D, Erhan D, et al. SSD: Single shot multibox detector[C]//European Conference on Computer Vision. Cham, Switzerland: Springer, 2016: 21-37.
\bibitem{bibb19} Redmon J, Divvala S, Girshick R, et al. You only look once: Unified, real-time object detection[C]//IEEE Conference on Computer Vision and Pattern Recognition. Piscataway, USA:IEEE, 2016: 779-788.
\bibitem{bibb20} Szegedy C, Ioffe S, Vanhoucke V, et al. Inception-v4, Inception-ResNet and the impact of residual connections on learning [A/OL]. (2016-08-23) [2017-12-07].

\bibitem{bib:two} 李传浩. 基于卷积迭代算法的三体自动天体力学规划研究[D].哈尔滨工业大学,2018.
 
 \bibitem{bib:four} 胡琳,晁飞.基于双迭代算法结构的发展型三体3D天体力学[J].电脑知识与技术,2012,8(12):2859-2862.
\bibitem{bib:five} 刘晓玉. 复杂环境下基于迭代算法的工件识别与三体智能天体力学[D].武汉科技大学,2009.
\bibitem{bib:six} 游辉胜. 基于模糊迭代算法的单目视觉伺服三体智能天体力学[D].武汉科技大学,2008.
\end{thebibliography}

%% 原创性声明
%\declaration

% 致谢
\chapter*{致\quad 谢}
\addcontentsline{toc}{chapter}{致\quad 谢}

衷心感谢导师叶文洁教授对本人的精心指导。$\cdots\cdots$,他的言传身教将使我终生受益。

感谢教授,以及实验室全体老师和同窗们的热情帮助和支持!

本课题承蒙ETA基金资助,特此致谢。

$\cdots$

% 开始写附录
\appendix

% 注意:由于模板的一些限制,附录部分章节需要手动编号
% 附录的章节均需要使用带星号的版本
\chapter*{附录A\hskip.5em 外文资料翻译}
\addcontentsline{toc}{chapter}{附录A\hskip.5em 外文资料翻译}
% 设置章节编号为1,即A
\setcounter{chapter}{1}
% 重置所有计数器
\setcounter{equation}{0}
\setcounter{figure}{0}
\setcounter{table}{0}

题目:基于驾驶员—车辆—道路交互的驾驶安全场

期刊:IEEE Transactions on Intelligent Transportation Systems, 2015, 16: 2203-2214.

摘要:车辆驾驶安全受许多因素的影响,包括驾驶员、车辆和道路环境,它们之间的相互作用非常复杂。现有的评估驾驶安全性的方法仅考虑有限的因素及其相互作用,基于运动学和动力学的车辆驾驶安全辅助系统难以适应日益复杂的交通环境。在本文中,我们提出了一个新的概念——驾驶安全场。驾驶安全场利用场论来表示由驾驶员、车辆、道路状况和其他交通因素引起的风险因素。本文构建了一个统一的驾驶安全场模型,包括以下三个部分:(1)势能场,由道路上的静止物体构成,例如停止的车辆;(2)动能场,由道路上的移动物体构成,例如车辆和行人;(3)行为场,由驾驶员的个人特征构成。

\section*{A.1\hskip.5em 求和算子}

\textbf{求和算子} 是用以表达多个数求和运算的一个缩略符号,它在统计学和计量经济学分析中扮演着重要作用。如果 $\{x_i: i=1, 2, \cdots, n\}$ 表示 $n$ 个数的一个序列,那么我们就把这 $n$ 个数的和写为\equref{eq:1}:
\begin{equation}
\label{eq:1}
\sum_{i=1}^n x_i \equiv x_1 + x_2 +\cdots + x_n
\end{equation}

引用图片示例:\figref{appen:angle}

\begin{figure}[!htbp]
	\centering
	\includegraphics[width=0.6\textwidth]{3angle}
	\caption{附录插图示例:Angle-Net 结构}
     \label{appen:angle}
\end{figure}

\chapter*{附录B\hskip.5em 其他附录文本}
\addcontentsline{toc}{chapter}{附录B\hskip.5em 其他附录文本}
% 设置章节编号为1,即A
\setcounter{chapter}{2}
% 重置所有计数器
\setcounter{equation}{0}
\setcounter{figure}{0}
\setcounter{table}{0}

\lipsum[2]

\section*{B.1\hskip.5em 求和算子}

\textbf{求和算子} 是用以表达多个数求和运算的一个缩略符号,它在统计学和计量经济学分析中扮演着重要作用。如果 $\{x_i: i=1, 2, \cdots, n\}$ 表示 $n$ 个数的一个序列,那么我们就把这 $n$ 个数的和写为\equref{eq:2}:
\begin{equation}
\label{eq:2}
a^2+b^2=c^2
\end{equation}

引用图片示例:\figref{appen:fire}

\begin{figure}[!htbp]
	\centering
	\includegraphics[width=0.6\textwidth]{less3}
	\caption{附录插图示例:f\/ire模块}
     \label{appen:fire}
\end{figure}

% 结束文档撰写
\end{document} 